% Created 2015-07-29 Wed 19:34
\documentclass[11pt]{article}
\usepackage[utf8]{inputenc}
\usepackage[T1]{fontenc}
\usepackage{fixltx2e}
\usepackage{graphicx}
\usepackage{longtable}
\usepackage{float}
\usepackage{wrapfig}
\usepackage{rotating}
\usepackage[normalem]{ulem}
\usepackage{amsmath}
\usepackage{textcomp}
\usepackage{marvosym}
\usepackage{wasysym}
\usepackage{amssymb}
\usepackage{hyperref}
\tolerance=1000
\author{Albin Severinson}
\date{\today}
\title{GCDC16 Communication Architecture}
\hypersetup{
  pdfkeywords={},
  pdfsubject={},
  pdfcreator={Emacs 24.3.1 (Org mode 8.2.10)}}
\begin{document}

\maketitle
\tableofcontents

\begin{center}
\begin{tabular}{lr}
Version: & 1.0\\
 & \\
 & \\
\end{tabular}
\end{center}

\newpage
\section{Introduction}
\label{sec-1}
This document will provide an overview of the communication stack that
will be used for GCDC16. It will also provide an introduction to the
message sets used, and to how the sensor fusion system and the
scenario-specific Simulink models will communicate with other
vehicles. This document will be updated during the project.

\section{Message Sets}
\label{sec-2}
This section breifly explains the message sets used for GCDC. Another
document will be released to cover the messages in more detail.

\subsection{CAM}
\label{sec-2-1}
Cooperative awareness messages (CAM) is the "hello" message. It is
broadcasted at a rate of up to 10Hz and contains detailed information
about the vehicle broadcasting it. There are methods for preventing
traffic congestion by, for example, lowering the message rate when the
vehicle is moving slowly. However, GCDC16 may require a non-standard
CAM transmission interval and may or may not use the methods for
preventing traffic congestion. A CAM message consists of a
header and up to four containers.

\begin{itemize}
\item The basic container includes basic information related to the
originating ITS station (ITS-S).
\item The high frequency container contains highly dynamic information
about the originating ITS-S, such as heading and speed.
\item The low frequency container contains static and not highly dynamic
information of the originating ITS-S.
\item The special vehicle container contains information specific to the
vehicle role of the originating ITS-S. This container is not needed
within i-GAME.
\end{itemize}

\emph{Source: Deliverable 3.2}

\subsection{DENM}
\label{sec-2-2}
Decentralized environmental notification message (DENM) is the message
used for event-based communication. It is sent to communicate with or
to notify other vehicles about some event and is sent to a all
vehicles withing a specific geographical area. A DENM message consists
of a header and up to four containers.

\begin{itemize}
\item The management container contains information related to the DENM
management and the DENM protocol.
\item The situation container contains information related to the type of
detected event. For example road work.
\item The location container contains information about the event location.
\item The à la carte container contains information specific to the event.
This container is used for cases where specific information not
included in the other containers needs to be transmitted.
\end{itemize}

\emph{Source: Deliverable 3.2}

\subsection{iGAME}
\label{sec-2-3}
The CAM and DENM message sets are developed primarily with platooning
in mind. To enable the complex automated driving required for GCDC,
the CEM and DENM message sets will be extended with a message set
specific for iGAME. This set is not yet defined, but it will include
messages for coordinating lane changes and such.

\emph{Source: Deliverable 3.2}

\section{Architecture}
\label{sec-3}
\includegraphics[width=.9\linewidth]{./img/com_arch_1.jpg}

\subsection{Antennae (OSI 1)}
\label{sec-3-1}
Not decided as of writing this, but must follow the ITS-G5 spec for
wireless transmission.

\subsection{Driver (OSI 2)}
\label{sec-3-2}
The network driver used. Not decided as of writing this.

\subsection{Network layer (OSI 3)}
\label{sec-3-3}
Will be part of the C-ITS stack or will run as a separate program.
Note that C-ITS in general does very little tunelling. Messages are
sent as datagrams that doesn't guarantee arrival and message order.
Instead the ITS stack is made to be resilient against message loss.

\subsection{C-ITS Stack}
\label{sec-3-4}
Cooperative intelligent transportation system software stack handles
creation of CAM and DENM messages. It will add header information and
package the data to follow specification.

\subsection{CAM service}
\label{sec-3-5}
The CAM service handles sending CAM messages at proper intervals. The
CAM service will listen for UDP message from the sensor fusion system
and will forward them to the C-ITS stack, which will transmit them. If
methods for preventing traffic congestion are used, the CAM service
will make a decision on whether or not to forward every time a message
is received.

The CAM service will expect UDP messages from the sensor fusion system
containing all information required for CAM messages at whatever rate
that is decided on. Currently, the highest rate of any message is
25Hz.


\subsection{DENM/iGAME Router}
\label{sec-3-6}
The DENM/iGAME router listens for messages coming to and from the
Simulink models controlling the car and is completely stateless as it
only forwards messages.

The router will expect UDP packets from the Simulink models
controlling the car, containing the message ID as defined in
deliverable 3.2 and additional information required to create the
specified DENM/iGAME message. Incoming messages will be forwarded to
the Simulink models that use that message.


\section{Internal communication}
\label{sec-4}
Communication between systems in the car will be performed using UDP
packets sent on the local LAN.

\subsection{Sensor Fusion}
\label{sec-4-1}
The CAM service will expect UDP messages from the sensor fusion system
containing all information required to create any CAM message. This
packet will be sent to the CAM service at the rate of the most
frequent CAM message. Currently, the highest rate of any message is
25Hz.

\subsection{Scenario-specific models}
\label{sec-4-2}
The scenario-specific models will communicate with the DENM router to
send and receive DENM/iGAME messages required for the scenario. For
example, pairing with other vehicles during the platoon merge
scenario.

The messages will include the message ID as specified in deliverable
3.2 and an alacarte container that is defined on a per-message basis.
For example, sending a message to pair with another vehicle during the
platoon marging scenario requires sending a UDP packet containing the
message ID for the pairing message and the vehicle ID of the vehicle
to pair with.
% Emacs 24.3.1 (Org mode 8.2.10)
\end{document}

% Created 2015-10-22 Thu 16:41
\documentclass[11pt]{article}
\usepackage[utf8]{inputenc}
\usepackage[T1]{fontenc}
\usepackage{fixltx2e}
\usepackage{graphicx}
\usepackage{grffile}
\usepackage{longtable}
\usepackage{wrapfig}
\usepackage{rotating}
\usepackage[normalem]{ulem}
\usepackage{amsmath}
\usepackage{textcomp}
\usepackage{amssymb}
\usepackage{capt-of}
\usepackage{hyperref}
\author{Albin Severinson}
\date{\today}
\title{GCDC16 Communication Bus}
\hypersetup{
 pdfauthor={Albin Severinson},
 pdftitle={GCDC16 Communication Bus},
 pdfkeywords={},
 pdfsubject={},
 pdfcreator={Emacs 24.4.1 (Org mode 8.3.1)}, 
 pdflang={English}}
\begin{document}

\maketitle
\tableofcontents

\begin{center}
\begin{tabular}{lrl}
Version & 0.1 & First draft\\
 & 0.2 & Added App bus\\
 &  & \\
\end{tabular}
\end{center}

\section{Introduction}
\label{sec:orgheadline1}
This document specifies the busses that connect to the communication
subsystem in the Simulink model. 4 bytes correspond to a Java integer,
and 8 bytes to a Java long.

For incoming messages, data elements not present in the message will
correspond to a NaN value on the bus.


\section{Sensor Fusion Bus}
\label{sec:orgheadline2}
This bus contains all data related to the sensor fusion subsystem.
There will be two identical buses, one from the sensor fusion system
to the communication, and one in the other direction. 

\begin{center}
\begin{tabular}{lrl}
Data: & Bytes: & Notes:\\
\hline
Station ID & 4 & Unique station ID, given by organisers\\
GenerationDeltaTime & 4 & See D3.2\\
StationType & 4 & \(=5\)\\
Latitude & 4 & See D3.2\\
Longitude & 4 & See D3.2\\
SemiMajorConfidence & 4 & See D3.2\\
SemiMinorConfidence & 4 & See D3.2\\
SemiMajorOrientation & 4 & See D3.2\\
Altitude & 4 & Not in D3.2?\\
Heading & 4 & See D3.2\\
HeadingConfidence & 4 & See D3.2\\
Speed & 4 & See D3.2\\
SpeedConfidence & 4 & See D3.2\\
VehicleLength & 4 & See D3.2\\
VehicleWidth & 4 & See D3.2\\
LongAcceleration & 4 & See D3.2\\
LongAccelerationConfidence & 4 & See D3.2\\
YawRate & 4 & See D3.2\\
YawRateConfidence & 4 & See D3.2\\
VehicleRole & 4 & \(=0\)\\
ReferenceTime & 8 & See D3.2\\
RearAxleLocation & 4 & See D3.2\\
ResponseTimeConstant & 4 & See D3.2, use "unavailable" value\\
ResponseTimeDelay & 4 & See D3.2, use "unavailable" value\\
 &  & \\
\end{tabular}
\end{center}



\section{Application Bus}
\label{sec:orgheadline3}
There are two busses connecting the communication system to the
application running the current scenario:
\begin{itemize}
\item Application Data Bus
\item Application Control Bus
\end{itemize}

The application data bus contains all data needed for communication.
The application control bus is used to instruct the communication
system to create a specific message.

In order to send a specific message, the following steps
must be performed:
\begin{enumerate}
\item Make sure all required data for the message is present on the
application data bus.
\item Set the correct value on the application control bus for the
desired message.
\end{enumerate}

\begin{center}
\begin{tabular}{llrl}
Data: & Scenario: & Bytes: & Notes:\\
\hline
DetectionTime &  & 8 & See D3.2\\
CauseCode &  & 4 & See D3.2\\
SubCauseCode &  & 4 & See D3.2\\
Controller type &  & 4 & See D3.2\\
Target longitudinal acceleration &  & 4 & See D3.2\\
Time headway &  & 4 & See D3.2\\
Cruise speed &  & 4 & See D3.2\\
(opt) Participants ready &  & 4 & See D3.2\\
(opt) Start platoon &  & 4 & See D3.2\\
(opt) End-of-scenario &  & 4 & See D3.2\\
Mio ID &  & 4 & See D3.2\\
Mio Range &  & 4 & See D3.2\\
Mio Bearing &  & 4 & See D3.2\\
Mio Range rate &  & 4 & See D3.2\\
Lane &  & 4 & See D3.2\\
Forward ID &  & 4 & See D3.2\\
Backward ID &  & 4 & See D3.2\\
Merge request &  & 4 & See D3.2\\
Safe-to-merge &  & 4 & See D3.2\\
Flag &  & 4 & See D3.2\\
Flag tail &  & 4 & See D3.2\\
Flag head &  & 4 & See D3.2\\
Platoon ID &  & 4 & See D3.2\\
Distance travelled in CZ &  & 4 & See D3.2\\
Intention &  & 4 & See D3.2\\
Counter &  & 4 & See D3.2\\
 &  &  & \\
\end{tabular}
\end{center}
\end{document}

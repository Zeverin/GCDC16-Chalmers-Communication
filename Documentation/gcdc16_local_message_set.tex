% Created 2015-08-15 Sat 17:30
\documentclass[11pt]{article}
\usepackage[utf8]{inputenc}
\usepackage[T1]{fontenc}
\usepackage{fixltx2e}
\usepackage{graphicx}
\usepackage{grffile}
\usepackage{longtable}
\usepackage{wrapfig}
\usepackage{rotating}
\usepackage[normalem]{ulem}
\usepackage{amsmath}
\usepackage{textcomp}
\usepackage{amssymb}
\usepackage{capt-of}
\usepackage{hyperref}
\author{Albin Severinson}
\date{\today}
\title{GCDC16 Local Message Set}
\hypersetup{
 pdfauthor={Albin Severinson},
 pdftitle={GCDC16 Local Message Set},
 pdfkeywords={},
 pdfsubject={},
 pdfcreator={Emacs 24.4.1 (Org mode 8.3.1)}, 
 pdflang={English}}
\begin{document}

\maketitle
\tableofcontents

\begin{center}
\begin{tabular}{lrl}
Version: & 0.1 & First draft\\
 &  & \\
 &  & \\
\end{tabular}
\end{center}

\newpage
\section{Introduction}
\label{sec:orgheadline1}
This document presents the local message set (LMS) to be used for
GCDC16. LMS will be used by the sensor fusion system to generate CAM
messages, and by the scenario control models to generate DENM/iGAME
messages.

The communication stack includes a vehicle adapter that will receive
these messages and use them to create proper CAM/DENM/iGAME messages
that will be forwarded to other vehicles. The LMS follows the ETSI
specification as closely as possible, but makes some changes to make
is possible to create the messages in Simulink.

Different network ports will be used for CAM/DENM/iGAME messages in
order to distinguish them.

All data is in network byte order, which is identical to big endian.

\section{CAM}
\label{sec:orgheadline10}
\begin{itemize}
\item How is time measured? How is an Instant defined?
\item Define:
\begin{itemize}
\item Time, Instant
\item curvature
\item yawRate
\item headingDegreesFromNorth
\end{itemize}
\end{itemize}

CAM consists of a single large message that is sent to the
communication stack periodically. The communication stack will pick up
the message and make a decision on what parts of the message to forward
to other vehicles. The message should contain the specified data, in
the specified order. Data marked as N/A should be all zeroes.

\begin{center}
\begin{tabular}{lll}
\hline
Datatype: & Data: & Notes:\\
\hline
int & curvature & How is this defined?\\
byte & curvatureConfidence & \\
byte & accelerationControlStatus & Details below\\
byte & exteriorLightsStatus & Details below\\
byte & driveDirection & forward=0, backward=1, unavailable=2\\
int & yawRate & How is this defined?\\
byte & yawRateConfidence & \\
byte & stationType & 5 for passenger cars\\
byte & vehicleRole, & specialTransport=2, default=0\\
boolean & embarkationStatus & N/A\\
byte & dangerousGoods & N/A\\
byte & dangerousGoodExt & N/A\\
byte & lightBarSiren & N/A\\
byte & ptActivationType & N/A\\
byte[] & ptActivationData & N/A, How many bytes?\\
byte[] & longPositionVector & Details below\\
 &  & \\
\end{tabular}
\end{center}

Listed as unavailable in GeoNetworking stack:
\begin{itemize}
\item SemiAxisLength
\item HeadingValue
\item AltitudeValue
\item AltitudeConfidence
\item HeadingConfidence
\item SpeedConfidence
\end{itemize}

Spec. according to D3.2
\begin{itemize}
\item What happened to curvature?
\end{itemize}
\begin{center}
\begin{tabular}{rll}
Bytes: & Data: & Notes\\
\hline
1 & header & \\
4 & GenerationDeltaTime & \\
4 & Station ID & \\
1 & Station Type & \\
1 & Vehicle Role & \\
2 & Vehicle Length & \\
2 & Vehicle Width & \\
0 & Reference position & \\
4 & Latitude & \\
4 & Longitude & \\
? & Position Confidence Ellipse 95\% & Need more details\\
? & Altitude & Not in D3.2?\\
2 & Heading & \\
1 & Heading confidence 95\% & \\
2 & Speed & \\
1 & Speed Confidence 95\% & \\
2 & Yaw Rate & \\
1 & Yaw Rate Confidence 95\% & \\
2 & Longitudinal vehicle acceleration & \\
1 & Longitudinal vehicle acceleration confidence 95\% & \\
\end{tabular}
\end{center}



\subsection{accelerationControlStatus}
\label{sec:orgheadline2}
\begin{center}
\begin{tabular}{rl}
\hline
Bit: & Data:\\
\hline
0 & brakePedalEngaged\\
1 & gasPedalEngaged\\
2 & emergencyBrakeEngaged\\
3 & collisionWarningEngaged\\
4 & accEngaged\\
5 & cruiseControlEngaged\\
6 & speedLimiterEngaged\\
\end{tabular}
\end{center}

\subsection{exteriorLightsStatus}
\label{sec:orgheadline3}
\begin{center}
\begin{tabular}{rl}
\hline
Bit: & Data:\\
\hline
0 & lowBeamHeadlightsOn\\
1 & highBeamHeadlightsOn\\
2 & leftTurnSignalOn\\
3 & rightTurnSignalOn\\
4 & daytimeRunningLightsOn\\
5 & reverseLightOn\\
6 & fogLightOn\\
7 & parkingLightsOn\\
\end{tabular}
\end{center}

\subsection{longPositionVector}
\label{sec:orgheadline9}
\begin{center}
\begin{tabular}{rlll}
\hline
Byte: & Datatype: & Data: & Notes:\\
\hline
0-8 & Address & address & Details below\\
9-12 & Instant & timestamp & Details below\\
13-20 & Position & position & Details below\\
21-22 & short & confidenceAndSpeed & Details below\\
23-24 & short & headingDegreesFromNorth & Details below\\
 &  &  & \\
\end{tabular}
\end{center}


\subsubsection{address}
\label{sec:orgheadline4}
\begin{center}
\begin{tabular}{rll}
\hline
Bit: & Data: & Notes:\\
\hline
63 & isManual & Should be 1\\
62-58 & stationType & 5 for passenger cars\\
57-48 & countryCode & Haven't found Sweden\\
47-0 & lowLevelAddress & Unique station address\\
 &  & \\
\end{tabular}
\end{center}

\subsubsection{timestamp}
\label{sec:orgheadline5}
Time according to the TAI spec. From
\url{http://stjarnhimlen.se/comp/time.html}:
TAI = International Atomic Time (Temps Atomique International = TAI) is
     defined as the weighted average of the time kept by about 200
     atomic clocks in over 50 national laboratories worldwide.
     TAI-UT1 was approximately 0 on 1958 Jan 1.

GPS time = TAI - 19 seconds

This is to account for leap seconds, which are not added to GPS time.
The time is sent as an unsigned 32-bit integer.

\subsubsection{position}
\label{sec:orgheadline6}
\begin{center}
\begin{tabular}{ll}
\hline
Type: & Data:\\
\hline
int & latitudeDegrees\\
int & longitudeDegrees\\
\end{tabular}
\end{center}

\subsubsection{confidenceAndSpeed}
\label{sec:orgheadline7}
\begin{center}
\begin{tabular}{rll}
\hline
Bit: & Data: & Notes:\\
\hline
0-14 & speed & Signed units of speed, in 0.01 meters per second\\
15 & position accuracy indicator & 1 if position is confident and 0 otherwise. When is it confident?\\
 &  & \\
\end{tabular}
\end{center}

\subsubsection{headingDegreesFromNorth}
\label{sec:orgheadline8}
Heading is sent as an unsigned units of 0.1 degrees from North.

\section{DENM}
\label{sec:orgheadline11}

\begin{center}
\begin{tabular}{rlrl}
ID: & Message: & Bytes: & Data:\\
\hline
38 & Event Type & 1 & ID\\
 &  & 8 & Timestamp\\
 &  & 1 & Cause Code\\
 &  & 1 & Sub Cause Code\\
\hline
39 & Closed Lanes & 1 & ID\\
 &  & 8 & Timestamp\\
 &  & 1 & Driving Lane Status\\
\hline
40 & Lane Position & 1 & ID\\
 &  & 8 & Timestamp\\
 &  & 1 & Lane Position\\
 &  &  & \\
\end{tabular}
\end{center}


\section{iGAME}
\label{sec:orgheadline12}
The iGAME message set is still under proposal. Details on this set
will be presented in a future release of this document.
\end{document}
